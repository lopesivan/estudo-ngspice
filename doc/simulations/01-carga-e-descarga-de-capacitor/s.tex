\documentclass{article}
\usepackage{amsmath}

\begin{document}

\section*{Cálculo do Tempo de Carga de um Capacitor em um Circuito RC}

A constante de tempo $\tau$ para um circuito RC é dada pelo produto da resistência ($R$) e da capacitância ($C$). A fórmula para calcular a constante de tempo é:
\[
\tau = R \times C
\]
onde $R$ é a resistência e $C$ é a capacitância do capacitor.

\subsection*{Dados do Problema}
\begin{itemize}
    \item Resistência, $R = 1.2 \, \text{M}\Omega = 1.2 \times 10^6 \, \Omega$
    \item Capacitância, $C = 20 \, \mu\text{F} = 20 \times 10^{-6} \, \text{F}$
    \item Tensão inicial, $V_0 = 25 \, \text{V}$
\end{itemize}

\subsection*{Cálculo da Constante de Tempo}
Substituindo os valores fornecidos na fórmula, temos:
\[
\tau = (1.2 \times 10^6 \, \Omega) \times (20 \times 10^{-6} \, \text{F}) = 24 \, \text{s}
\]

\subsection*{Tempo de Carga}
O tempo necessário para que o capacitor carregue até um certo percentual da tensão de alimentação pode ser estimado utilizando múltiplos da constante de tempo $\tau$. Para carregar até aproximadamente:
\begin{itemize}
    \item 95\% da tensão máxima: $t \approx 3\tau$
    \item 99\% da tensão máxima: $t \approx 5\tau$
\end{itemize}

\subsubsection*{Cálculos para Percentuais Específicos}
\begin{align*}
t_{95\%} &\approx 3 \times 24 \, \text{s} = 72 \, \text{s} \\
t_{99\%} &\approx 5 \times 24 \, \text{s} = 120 \, \text{s}
\end{align*}

Estes tempos fornecem uma estimativa de quanto tempo o capacitor no circuito RC levará para alcançar 95\% e 99\% da tensão de carga total, respectivamente.

\end{document}
